
\documentclass[a4paper,12pt]{article}

\usepackage{amsmath}
\usepackage{amssymb}
\usepackage{amsthm}
\usepackage{amsfonts}
\usepackage{changepage}
\usepackage{mathtools}
\usepackage{booktabs}
\usepackage{caption}

\newcommand{\sumn}{\sum_{i=1}^{n}}
\newcommand{\sumt}{\sum_{t=0}^\infty}
\newcommand{\sumj}{\sum_{j=-\infty}^\infty}
\newcommand{\fsum}{\frac{1}{n} \sum_{i=1}^n}
\newcommand{\prodn}{\prod_{i=1}^{n}}
\newcommand{\intf}{\int_{-\infty}^{\infty}}
\newcommand{\intz}{\int_0^\infty}
\newcommand{\limf}{\lim_{n\to \infty}}
\newcommand{\N}{\mathbb{N}}
\newcommand{\R}{\mathbb{R}}
\newcommand{\E}{\mathbb{E}}
\newcommand{\Z}{\mathbb{Z}}
\newcommand{\C}{\mathbb{C}}
\newcommand{\bp}{\mathbb{P}}
\newcommand{\F}{\mathcal{F}}
\newcommand{\Var}{\text{Var}}
\newcommand{\Cov}{\text{Cov}}
\newcommand{\Corr}{\text{Corr}}
\newcommand{\topr}{\xrightarrow{  p  }}
\newcommand{\tod}{\xrightarrow{  d  }}
\newcommand{\blambda}{\bar{\lambda}}
\newcommand{\htheta}{\hat{\theta}}
\newcommand{\hbeta}{\hat{\beta}}
\newcommand{\hmu}{\hat{\mu}}
\newcommand{\hF}{\hat{F}}
\newcommand{\sss}{\subsubsection*}
\newcommand{\simiid}{\stackrel{\text{iid}}{\sim}}
\newcommand{\eqas}{\stackrel{\text{a.s.}}{=}}
\newcommand{\eps}{\varepsilon}
\newcommand{\re}{\text{Re}}
\newcommand{\im}{\text{Im}}

\numberwithin{equation}{section}

\theoremstyle{definition}
\newtheorem{thm}{Theorem}
\newtheorem{claim}{Claim}
\newtheorem{prop}[thm]{Proposition}
\newtheorem{defn}{Definition}
\newtheorem{cor}{Corollary}
\newtheorem{ex}{Example}
\newtheorem{exer}{Exercise}
\newtheorem{lem}[thm]{Lemma}
\newtheorem{ob}{Observation}
\newtheorem{fact}{Fact}

\allowdisplaybreaks[1]
\linespread{1.6}
% \linespread{1.3}

\begin{document}

\small
% \footnotesize

\setlength\voffset{-0.75 in}

\changepage{1.5 in}{1 in}{0 in}{-0.5 in}{0 in}{0 in}{0 in}{0 in}{0 in}

% Description of the problem

\section{Description of the Problem}

\subsection{Introduction to Macroeconomics}

Macroeconomics is the subfield of economics that deals with outcomes
for an entire economy, as opposed to a single market. A standard
macroeconomic model will seek to describe an economy in which prices
adjust so that demand equals supply, and such that the interest rates
on financial assets adjust so that the quantity of financial assets
sold equals the quantity of financial assets purchased.

For several decades, the trend has been to develop ``microfounded''
models, in which macro-level behavior like total consumption is
derived by considering the consumption decision of individual economic
agents, and then aggregating over many agents' actions to obtain an
overall result. These microfoundations are designed to keep
macroeconomic models closer in line with reality, and provide testable
checks on the theories macroeconomics propose (i.e., if your model is
correct, then X, Y, and Z should be observed in micro-level data).

However, the task of aggregating over many individual decisions can
impose serious mathematical difficulties. In particular, macroeconomic
outcomes may depend on the entire distribution of individual states
and characteristics across individuals. For example, the behavior of
an economy with a large degree of wealth inequality may differ from
one with less inequality, even if the two economies exhibit the same
average level of wealth. Therefore, modeling the macroeconomy using a
microfounded model may depend on keeping track of entire distributions
for each variable --- infinite-dimensional objects that are difficult
to work with numerically.

These obstacles have often been overcome through the use of
simplifying, but unrealistic, assumptions, to ensure that the
distributions of states across agents do not matter, and that the
overall state of the macroeconomy can be summarized in a few aggregate
statistics. These assumptions often take the form of perfect insurance
markets, in which agents can insure against any possible event that
may occur. Since agents do not like risk, they generally insure away
all their individual risk, so that their behavior only depends on
aggregate conditions --- allowing for easy aggregation.

\subsection{The Heterogeneous-Agent Approach}

While these assumptions have allowed for many years of productive
macroeconomic research, they miss major features of the choices facing
most individuals. In reality, people face many forms of uninsurable
risk such as unexpected changes to wages or unemployment. In addition,
most people only have access to a limited set of financial instruments
with which to invest, and often face strict borrowing limits,
especially on unsecured debt, which they cannot exceed.

Incorporating these features into a microfounded macroeconomic model
leads to what is typically known as a ``heterogeneous-agent'' model,
in which micro-level differences between agents are important, and the
entire distribution of individual states must be accounted for. This
leads to three kinds of problems, all of which are typically
computationally intensive to overcome. 

\subsubsection{Solution}

The first issue is that removing perfect insurance markets yields a
much more complicated individual problem, in which agents must
carefully consider the risks posed by uninsurable fluctuations at all
times in the future. The optimal policies typically can only be
calculated numerically, and standard grid-based approximations of
policy functions are subject to the ``curse of dimensionality'' as the
number of state variables increases, leading to large computational
burden for all but the most simple models.

\subsubsection{Simulation}

The second problem posed by a heterogeneous-agent model is that
macroeconomic behavior now depends on the entire distribution of
individual states across agents. Therefore, analyzing the behavior
implied by the model typically involves simulating the behavior of
thousands of agents, and mechanically aggregating to obtain
macroeconomic results over thousands of time periods. If prices must
be set so that markets in goods or financial assets must clear, then
each period may need to be simulated many times as an algorithm finds
the correct price. Therefore, simulation can be a computationally
intensive step even if the underlying model is very simple and easy to
solve.

\subsubsection{Forecasting}

The final problem is that the agents in a
realistic model are forward looking, meaning that agents have to have
some way of forecasting what they expect to occur in the future based
on current conditions. When current conditions are determined by an
antire distribution across agents, then it is a challenging task to
translate that object into a reasonable forecast. Instead, economists
usually assume that agents use some forecasting rule based on
aggregate variables. But to be realistic, these rules can lead to
forecasting error, but should not be particularly biased
(i.e. everyone should not always forecast incorrectly in the same
way). But since behavior depends on the forecasting rule, and the bias
of the forecasting rule depends on behavior, this leads to a
``fixed-point'' problem that may involve running the solution and
simulation steps multiple times, adding to the computational burden.

\subsubsection{Computational Performance}

Despite these challenges, many of the computations involved in solving these
models are highly parallelizable, and should therefore yield massive
speedup relative to a serial computation. In particular, solving and
simulating these models often requires the type of repeated simple
calculation that is perfect for GPU computation. This should allow not
only for convenience allowed by the reduction in wait times, but much
more important, allow for more complex models or more accurate
solutions to become tractable.

\section{The Model}

This next section will summarize the economic model that is being solved, which will be the foundation for the computational routine. The reader interested in the computational routine only should skip ahead to Section \ref{sec:optim-policy-algor}.

\subsection{Environment}

Time is modeled as a sequence of discrete periods, indexed by $t$
(although I will typically suppress the dependence of variables on $t$
in the notation). The economy is populated by a continuum of
infinitely-lived agents indexed by $i$. Agents earn labor income based
on their own individual employment state, and the overall state of the
macroeconomy. The employment state for agent $i$ is denoted $e_i$,
with $e_i = 1$ when the agent is employed, and $e_i = 0$ when the
agent is unemployed. The macroeconomic state is denoted $z$, with $z =
0$ when the economy is in a low-productivity (``recession'') state,
and $z = 1$ when the economy is in a high-productivity (``expansion'')
state. Define $s_i = (z, e_i)$ to be the overall state for agent $i$,
which can take on four possible values, and let $y(s_i)$ be the
function (identical across individuals) that translates the states
into an individual's labor income. We will assume that $s_i$ follows a
Markov chain for each agent with transition matrix $P$. 

In practice, we parameterized the $z$ values, the $P$ matrix, and the $y$ function following Krusell and Smith (1998), which in turn chooses these values to match various empirical features of the macroeconomy. Unlike Krusell and Smith, however, I chose the $y$ function so that agents receive 1/3 of their employed income in the unemployed state, which roughly corresponds to U.S. unemployment insurance. 

\subsection{Preferences}

In each period, agents consume resources. Agents choose
state-contingent paths of lifetime consumption in each period $t$ to
maximize
\begin{equation*} 
  V_{it} = \E_t \sum_{j=0}^\infty \beta^j u(c_{i,t+j}) 
\end{equation*} 
where $V$ is the present discounted value of lifetime consumption, $\E_t$ is the mathematical expectations operator conditional on time $t$ information, $\beta$ is the discount factor (which determines the agent's level of patience), $u$ is a utility funciton representing the benefit an agent gets from some level of consumption in a given period, and $c$ is consumption. Further, $V$ and $c$ should be interpreted as functions that may depend on the state at time $t$ and $t+j$, respectively. We will restrict attention to problems where $u$ is continuously differentiable and strictly concave.

\subsection{Assets}

Agents can save and borrow from each other by holding positive and
negative positions in a one-period riskless bond. Denote agent $i$'s
holdings of the bond by $b_i$. Each agent can buy (or sell) one unit
of the bond at price $q$, in which case they receive (or pay) one unit
of consumption in the following period. This is equivalent to an
interest rate of $r = 1/q$. Since there is no outside source of funds,
in each period $q$ must be set so that total saving equals total
borrowing, which is known as ``market clearing.'' Mathematically, this
can be expressed by the condition
\begin{equation*} 
\int b_i \,di = 0 
\end{equation*}
which must hold in each period.

\subsection{Agent's Problem}

Each agent enters a given period with wealth $x$, where $x$ is the
amount of the consumption good that the agent is due from his or her
previous bond purchases ($x_{it} = b_{i,t-1}/q_{t-1}$). Each agent
solves the problem
\begin{align*}
  V(x_i, q, s_i) &= \max_{c_i, b_i} u(c_i) + \beta \E \left[ V(b_i, q', s_i') \, \Bigr| s_i \right] \\
  \text{subject to}& \\
  c_i + q b_i &= x_i + y(s_i) \\
  c_i &\ge 0 \\
  b_i &\ge -B
\end{align*}
where primes (i.e. $x_i'$) represent values of variables in the
following period.\footnote{Hopefully this is not confusing when primes
  are also used for derivatives.}

Given our earlier assumptions on the utility function, the agent's optimal policy $c(x, q, s_i)$ is uniquely defined by the first order condition 
\begin{equation}
\label{eq:foc}
  q u'(c(x_i, q, s_i)) \ge \beta \E \left[ u'(c(x_i', q', s_i')) \Bigr| s_i \right]
\end{equation}
which must hold with equality for $b_i > -B$, where $b_i = x_i +
y(s_i) - c_i$. The goal of the solution algorithm will be to solve for this optimal policy function $c$ up to a close approximation.

\section{Optimal Policy Algorithm}
\label{sec:optim-policy-algor}

\subsection{Strategy and Approximation}

The first computational task is to find the unique function $c(x_i, q,
s_i)$ that solves the optimality condition \eqref{eq:foc}. In general,
the solution will proceed by starting with some guess of the policy
function $c^0$, and then on each step, choosing $c^{n+1}$ to satisfy
\begin{equation}
\label{eq:foc-iter}
  q u'(c^{n+1}(x_i, q, s_i)) \ge \beta \E \left[ u'(c^n(x_i', q', s_i')) \Bigr| s_i \right]
\end{equation}
for all $(x_i, q, s_i)$, which again must hold with equality for $b_i > 0$. 

Since $x_i$ and $q$ are continuous variables, the function $c$ is an
infinite-dimensional object, we must approximate it to be able to
solve numerically. For this project, we use a simple approximation by
choosing grids of points $(\bar{x}_1, \ldots, \bar{x}_{N_x})$ and
$(\bar{q}_1, \ldots, \bar{q}_{N_q})$, and approximate $c$ by defining
it at all combinations of these gridpoints and the state $s_i$. We can
then use bilinear interpolation over the $(x_i, q)$ dimension (holding
$s_i$ fixed) to evaluate the function.

For reasons that will become clear as the algorithm progresses, we
implement this algorithm in OpenCL using work-groups of size $(K_x, 1,
N_s)$, with $K_x \le N_x$.

\subsection{Endogenous Grid Method}
\label{sec:endog-grid-meth}

A simple way to perform an iteration of \eqref{eq:foc-iter} would be to
use a nonlinear equation solver for each point $(\bar{x}_j, \bar{q}_k,
\bar{s}_l)$ on the grids defined earlier. However, a nonlinear
equation solver will typically require many function evaluations to
find a solution, in addition to the added complexity of the inequality
constraint, making it a relatively slow and complicated method to
implement

A better method, assuming that the function $u'$ has a known inverse,
is to take advantage of the fact that the left side of \eqref{eq:foc} can
be inverted in closed form (although the right side cannot). Along
these lines, the improved method is to define a grid over bond
holdings, $(\bar{b}_0, \ldots, \bar{b}_{N_b-1}$, and solve for the
right hand side of \eqref{eq:foc} given $(\bar{b}_j, \bar{q}_k,
\bar{s}_l)$. To perform this task efficiently in parallel, each work
item with, say, global id $(p, 1, r)$ calculates $u'(\bar{b}_p,
\tilde{q}(\bar{s}_r), \bar{s}_r)$ and stores the resulting value in
local memory. Each work-item, again with arbitrary global id $(p, 1,
r)$, now sums over the terms $P(r, r') u'(\bar{b}_p,
\tilde{q}(\bar{s}_{r'}), \bar{s}_{r'})$ for $r' = 1, \ldots, N_s$,
using the previous results. This is the logic for choosing the $s$
dimension of each work-group to be equal to $N_s$.

With the expectation terms in hand on our $\bar{b}$ grid, we can then
solve for $c_i$ for each combination $(\bar{b}_j, \bar{q}_k,
\bar{s}_l)$ using the relation
\begin{equation*}
  c_i = u'^{-1} \left\{ \beta \bar{q}_k^{-1} \beta \E \left[ u'(c^n(\bar{b}_j, \tilde{q}(s_i'), s_i')) \Bigr| s_i \right] \right\}
\end{equation*}
which allows us to solve \eqref{eq:foc} in one step. Note that we do not
have to worry about \eqref{eq:foc} holding with equality because at most
one point on our $\bar{b}$ grid is equal to $-B$. Define $c_j^*$ to be
the value obtained by solving \eqref{eq:foc} for $b_i = \bar{b}_j$. Then
we can return to a mapping between $x_i$ and $c_i$ by using the
definition $x_i = c_i + b_i - y(s_i)$ to solve for the implied
starting wealth values $x_j^*$ given $c_j^*$ and $\bar{b}_j$.

Applying this algorithm on a given iteration involves many evaluations
of $c^n$ using bilinear interpolation. As long as we have chosen our
$\bar{x}$ and $\bar{q}$ grids such that the mappings between some
value $x_i$ and the nearest lower neighbor $\bar{x}_j$ can be easily
evaluated, the bilinear interpolation procedure involves only a few
floating point calculations, requires access to only four points of
data, and is in general not a major computational burden. In
particular, we used polynomially spaced grids
\begin{align*}
  \bar{x}_j = x_{min} + (x_{max} - x_{min}) \left( \frac{j}{N_x-1} \right)^{k_x}
  \bar{q}_j = q_{min} + (q_{maq} - q_{min}) \left( \frac{j}{N_q-1} \right)^{k_q}
\end{align*}
with $k_x = 0.4$ (more points for $x_i$ small) and $k_q = 1$ (evenly
spaced).

\subsection{Recovering Original Grid}

The method of Section \ref{sec:endog-grid-meth} established an
efficient way to solve \eqref{eq:foc} over a grid of values for bond
holdings $b_i$, establishing a mapping between $x_j^*$ and $c_j^*$ for
each $\bar{b}_j$. However, in order to be able to perform bilinear
interpolation in the next iteration, we need to return to a uniform
grid system for the $x_i$ variable that does not depend on $j$. In
particular, we will return to our original grid system via linear
interpolation.

For fixed $\bar{q}_k$ and $\bar{s}_l$, we want to be able to evaluate
$c^{n+1}(\bar{x}_j, \bar{q}_k, \bar{s}_l)$ for each point in our
$\bar{x}$ grid. We can do this by finding the relevant values of
$x_j^*$ such that $x_m^* \le \bar{x}_j \le x_{m+1}^*$, and then
interpolating accordingly between $c_m^*$ and $c_{m+1}^*$.

There are two related challenges involved in this operation. The first
is that the $x^*$ grid is unknown (since it is a product of the
algorithm) and has no known mapping to immediately obtain the correct
bin given the value $\bar{x}_j$. Therefore, we will have to search for
the correct bin before interpolating. The second challenge is that
each work-group holds only a subset of the points of the $x^*$ grid
for a given $(\bar{q}_k, \bar{s}_l)$ if the $x$ dimension is at all
large. Since we cannot synchronize across work-groups, a second
challenge is finding which work group should perform the interpolation
for each point in the $\bar{x}_i$ grid.

We solved these problems by having each work-group load every element
of the $\bar{x}$ grid, one $K_x \times 1$ sized block at a time. One
element of each block is assigned to each work-item, and that
work-item then checks whether that value of $\bar{x}_j$ falls in that
work-group's subset of $x^*$ points, which is easily done since the
$x_j^*$ points are monotonically increasing in $j$. If the value of
$\bar{x}_j$ is not within that subset, the work-item does nothing and
proceeds to load its entry from the next block. If the value of
$\bar{x}_j$ is within that subset, the work-item searches for the
relevant bin (in local memory) using a simply bisection algorithm, and
then performs the linear interpolation. Once this is done, the
work-item updates the relevant value of $c^{n+1}(\bar{x}_j, \bar{q}_k,
\bar{s}_l)$ in global memory, and moves on to the next block.

The final complication is that points on the $\bar{x}$ grid may fall
below the bottom of the $x^*$ grid. If we set $\bar{b}_0 = -B$, as we
do in practice, then we know that these points must be in the
constrained region with $b_i = -B$, and can correspondingly set $c_i =
x_i + y(s_i) + B$.

\subsection{Summary}

This completes the algorithm to move from $c^n$ to $c^{n+1}$. To
summarize, the algorithm proceeds in the following steps.

\begin{enumerate}
\item Evaluate $\E_t u'(c^n(\bar{b}_j, \tilde{q}(\bar{s}_l),
  \bar{s}_l))$ forreach $(j,l)$.
\item Invert \eqref{eq:foc} to construct a $(x_j^*, c_j^*)$ mapping
  conditional on $\bar{b}_j$.
\item Load $\bar{x}$ block-by-block into each work-group, checking if
  points fall in the relevant $x^*$ values, and interpolating if so
  to recover $c^{n+1}$.
\end{enumerate}

The algorithm proceeds until the maximum distance between elements of
$c^n$ and $c^{n+1}$ falls below some tolerance.

\subsection{Implementation in OpenCL}

Most of the specifics of how each work-group and each work-item is assigned has already been described in the preceding sections, so this section will focus on performance and parallelization issues.

It should be clear that all steps of this algorithm can be run in
parallel, with only a few local synchronizations required. However,
because of the need to consider the entire list of $\bar{x}$ points in
each work-group, this algorithm will not scale linearly, and will lose
efficiency the larger the $\bar{x}$ grid is compared to the local size
$K_x$. In practice, we found that even for relatively large values of
$N_x$, the cost of checking irrelevant values of the $\bar{x}$ grid
appears tolerable. A further challenge is that depending on where
points on the $\bar{x}$ grid are concentrated relative to the $x^*$
grid, and so the work done by the various work groups in recovering
the original grid may be unbalanced, and it is possible that the
imbalance will increase with the number of $\bar{x}$ points. But
overall the algorithm is highly parallelizable, and should exhibit
massive speedup from running in parallel relative to in serial.

\section{Simulation Algorithm}
\label{sec:simulation-algorithm}

\subsection{Overall Strategy}

In order to determine macroeconomic behavior in a heterogeneous agent
model, simulation is required. While other methods exist, for example
keeping track of weights on a histogram, we use the method of simply
simulating the paths of a large number of agents. The main
computational challenge in this section is determining the bond price
in each period of the simulation that makes markets clear.

To initialize the simulation, we initialize agents with zero wealth,
and assign some random previous states to each agent and to the
overall economy. The initial condition can be relatively arbitrary, as
we will use a long ``burn in'' period for which we discard the
simulation results to ensure that the simulation has had time to lose
its dependence on initial conditions.

Once the simulation is initialized, we update the simulation
recursively, period-by-period. At the start of each period $t$, the
previous period's simulation gives us the starting wealth $x_{it}$ for
each agent. We then draw the overall state $z_t$ conditional on
$z_{t-1}$, and then draw $\eps_{it}$ for each $i$ conditional on
$\eps_{i,t-1}$, $z_{t-1}$, and $z_t$.

Given a guess for $q_t$, the bond price at time $t$, we evaluate
agents' optimal consumption using the $c$ function obtained from the
methods of Section \ref{sec:optim-policy-algor}. Given the resulting values of $c_{it}$, we can calculate the implied bond holdings using $b_{it} = x_{it} + y_{it} - c_{it}$, and sum over agents to get aggregate bond holdings, $\sum_i b_{it}$.

If this sum is within some tolerance of zero, we are done for this
period and move on. Otherwise, we adjust the guess of $q_t$ and repeat
the procedure until market clearing is obtained. In practice, we used
a bisection scheme to solve for the equilibrium value of $q_t$, using
the fact that $\sum_i b_{it} > 0$ when $q_t$ is too low, and $\sum_i
b_{it} < 0$ when $q_t$ is too high.

\subsection{Implementation in OpenCL}

To implement this in OpenCL, we assigned each work-item to calculate
the optimal policy for a single agent. This was performed using
bilinear interpolation as usual. However, since the number of bilinear
interpolation calculations over a large simulation of thousands of
agents over thousands of periods is likely to be large relative to the
number of gridpoints on which the $c$ function is defined, we
pre-calculated the interpolation coefficients over the entire grid on
the device using an OpenCL kernel.

The only other complication with implementation in OpenCL is the need
to evaluate the term $\sum_i b_{it}$, since the $i$ indices are
distributed across many work-groups. To address this, we used a
reduction method from the book \emph{OpenCL in Action}, which takes a
vector of entries in local memory and recursively adds the top half to
the bottom half to efficiencly add $b_{it}$ over each
work-group. Because there is no global synchronization, we then wrote
the sum for each work-group to global memory, and then called a second
kernel to perform the same reduction over the work-group sums, since
reasonable numbers of simulations do not require more than two
reductions.

Once the overall sum is calculated, it is transferred over to the
host, which then determines whether the sum is within the desired
tolerance, and if not, launches the entire routine again with a new
value of $q_t$. In this way, the algorithm only requires that a single
scalar be transferred between the host and device (once in each
direction) on each iteration.

\subsection{Summary}

Once the market clearing value $q_t$ is obtained, we then update time
$t+1$ starting wealth using $x_{i,t+1} = b_{it}$, and continue on in
the same fashion. The algorithm can therefore be summarized as follows.

\begin{enumerate}
\item Initialize agents starting wealth $x_{i0}$, previous employment states $\eps_{i,-1}$, and previous macro state $z_{-1}$, and set $t = 0$.
\item \label{item:1} Draw $z_t$ given $z_{t-1}$.
\item Draw $\eps_{it}$ given $z_t, z_{t-1}$ and $\eps_{i,t-1}$.
\item Initialize a guess for $q_t$.
\item \label{item:2} Calculate $c_{it}$ and $b_{it}$ for each agent using the optimal policy function.
\item If $\sum_i b_{it}$ is within tolerance of 0, proceed to Step \ref{item:3}, otherwise update the guess of $q_t$ and return to Step \ref{item:2}.
\item \label{item:3} Update $x_{i,t+1} = b_{it}$, increment $t = t+1$, and return to Step \ref{item:1}.
\end{enumerate}

The final step, once the simulated values have been
calculated for each $t$, is to discard some portion of the initial
observations as ``burn-in'' so that the resulting sample is not
dependent on initial conditions.

Overall, this algorithm is completely parallel when calculating
agents' policies, and highly parallel in the reduction step,although
some work items are idle at some times during the reduction. Further,
there is little overhead in terms of transferring data between host
and device until the computation is complete, and there are relatively
many floating point operations relative to memory operations, so
overall we expected substantial speedup from parallelizing this step
as well.

\section{``Meta'' Algorithm for Calculating $\tilde{q}$}

Throughout Section \ref{sec:optim-policy-algor} and Section
\ref{sec:simulation-algorithm}, we assumed a forecasting rule for
$\tilde{q}$, and the resulting optimal policy solution and simulation
depend on our guess for this rule. However, in principle there is good
reason to want this rule to be at least unbiased, so that agents'
estimates are not perpetually too low or too high.

Since our forecasting rule was based on the aggregate state $z$, this
is equivalent to demanding that $\tilde{q}(\bar{z}_j)$ be equal to the
sample average of values $q_t$ in periods in which $z_t =
\bar{z}_j$. To solve this fixed point problem, we begin with a guess
for $\tilde{q}$, and run the algorithms described in Section
\ref{sec:optim-policy-algor} and Section
\ref{sec:simulation-algorithm}. We then evaluate the sample means for each possible $z_t$ state, and update $\tilde{q}(\bar{z}_j)$ for the next iteration to be equal to this sample mean. With this new value of $\tilde{q}$ in hand, we proceed to the next iteration, and continue until the error between $\tilde{q}$ and the sample means falls within tolerance.

For this method, it is essential that the random draws of $\eps_{it}$ and $z_t$ be kept constant across iterations, otherwise the algorithm may not converge. For computational efficiency, it is important that each calculation of the optimal policy in Section \ref{sec:optim-policy-algor} begin with the solution from the previous iteration, which is likely to be very close to the new solution, and will substantially speed convergence time.

\end{document}
