
\documentclass[a4paper,12pt]{article}

\usepackage{amsmath}
\usepackage{amssymb}
\usepackage{amsthm}
\usepackage{amsfonts}
\usepackage{changepage}
\usepackage{mathtools}
\usepackage{booktabs}
\usepackage{caption}

\newcommand{\sumn}{\sum_{i=1}^{n}}
\newcommand{\sumt}{\sum_{t=0}^\infty}
\newcommand{\sumj}{\sum_{j=-\infty}^\infty}
\newcommand{\fsum}{\frac{1}{n} \sum_{i=1}^n}
\newcommand{\prodn}{\prod_{i=1}^{n}}
\newcommand{\intf}{\int_{-\infty}^{\infty}}
\newcommand{\intz}{\int_0^\infty}
\newcommand{\limf}{\lim_{n\to \infty}}
\newcommand{\N}{\mathbb{N}}
\newcommand{\R}{\mathbb{R}}
\newcommand{\E}{\mathbb{E}}
\newcommand{\Z}{\mathbb{Z}}
\newcommand{\C}{\mathbb{C}}
\newcommand{\bp}{\mathbb{P}}
\newcommand{\F}{\mathcal{F}}
\newcommand{\Var}{\text{Var}}
\newcommand{\Cov}{\text{Cov}}
\newcommand{\Corr}{\text{Corr}}
\newcommand{\topr}{\xrightarrow{  p  }}
\newcommand{\tod}{\xrightarrow{  d  }}
\newcommand{\blambda}{\bar{\lambda}}
\newcommand{\htheta}{\hat{\theta}}
\newcommand{\hbeta}{\hat{\beta}}
\newcommand{\hmu}{\hat{\mu}}
\newcommand{\hF}{\hat{F}}
\newcommand{\sss}{\subsubsection*}
\newcommand{\simiid}{\stackrel{\text{iid}}{\sim}}
\newcommand{\eqas}{\stackrel{\text{a.s.}}{=}}
\newcommand{\eps}{\varepsilon}
\newcommand{\re}{\text{Re}}
\newcommand{\im}{\text{Im}}

\allowdisplaybreaks[1]

\numberwithin{equation}{section}

\theoremstyle{definition}
\newtheorem{thm}{Theorem}
\newtheorem{claim}{Claim}
\newtheorem{prop}[thm]{Proposition}
\newtheorem{defn}{Definition}
\newtheorem{cor}{Corollary}
\newtheorem{ex}{Example}
\newtheorem{exer}{Exercise}
\newtheorem{lem}[thm]{Lemma}
\newtheorem{ob}{Observation}
\newtheorem{fact}{Fact}

\allowdisplaybreaks[1]
\linespread{1.6}
% \linespread{1.3}

\begin{document}

\small
% \footnotesize

\setlength\voffset{-0.75 in}

\changepage{1.5 in}{1 in}{0 in}{-0.5 in}{0 in}{0 in}{0 in}{0 in}{0 in}

% Description of the problem

\section{Description of the Problem}

\subsection{Introduction to Macroeconomics}

Macroeconomics is the subfield of economics that deals with outcomes
for an entire economy, as opposed to a single market. A standard
macroeconomic model will seek to describe an economy in which prices
adjust so that demand equals supply, and such that the interest rates
on financial assets adjust so that the quantity of financial assets
sold equals the quantity of financial assets purchased.

For several decades, the trend has been to develop ``microfounded''
models, in which macro-level behavior like total consumption is
derived by considering the consumption decision of individual economic
agents, and then aggregating over many agents' actions to obtain an
overall result. These microfoundations are designed to keep
macroeconomic models closer in line with reality, and provide testable
checks on the theories macroeconomics propose (i.e., if your model is
correct, then X, Y, and Z should be observed in micro-level data).

However, the task of aggregating over many individual decisions can
impose serious mathematical difficulties. In particular, macroeconomic
outcomes may depend on the entire distribution of individual states
and characteristics across individuals. For example, the behavior of
an economy with a large degree of wealth inequality may differ from
one with less inequality, even if the two economies exhibit the same
average level of wealth. Therefore, modeling the macroeconomy using a
microfounded model may depend on keeping track of entire distributions
for each variable --- infinite-dimensional objects that are difficult
to work with numerically.

These obstacles have often been overcome through the use of
simplifying, but unrealistic, assumptions, to ensure that the
distributions of states across agents do not matter, and that the
overall state of the macroeconomy can be summarized in a few aggregate
statistics. These assumptions often take the form of perfect insurance
markets, in which agents can insure against any possible event that
may occur. Since agents do not like risk, they generally insure away
all their individual risk, so that their behavior only depends on
aggregate conditions --- allowing for easy aggregation.

\subsection{The Heterogeneous-Agent Approach}

While these assumptions have allowed for many years of productive
macroeconomic research, they miss major features of the choices facing
most individuals. In reality, people face many forms of uninsurable
risk such as unexpected changes to wages or unemployment. In addition,
most people only have access to a limited set of financial instruments
with which to invest, and often face strict borrowing limits,
especially on unsecured debt, which they cannot exceed.

Incorporating these features into a microfounded macroeconomic model
leads to what is typically known as a ``heterogeneous-agent'' model,
in which micro-level differences between agents are important, and the
entire distribution of individual states must be accounted for. This
leads to three kinds of problems, all of which are typically
computationally intensive to overcome. 

\subsubsection{Solution}

The first issue is that removing perfect insurance markets yields a
much more complicated individual problem, in which agents must
carefully consider the risks posed by uninsurable fluctuations at all
times in the future. The optimal policies typically can only be
calculated numerically, and standard grid-based approximations of
policy functions are subject to the ``curse of dimensionality'' as the
number of state variables increases, leading to large computational
burden for all but the most simple models.

\subsubsection{Simulation}

The second problem posed by a heterogeneous-agent model is that
macroeconomic behavior now depends on the entire distribution of
individual states across agents. Therefore, analyzing the behavior
implied by the model typically involves simulating the behavior of
thousands of agents, and mechanically aggregating to obtain
macroeconomic results over thousands of time periods. If prices must
be set so that markets in goods or financial assets must clear, then
each period may need to be simulated many times as an algorithm finds
the correct price. Therefore, simulation can be a computationally
intensive step even if the underlying model is very simple and easy to
solve.

\subsubsection{Forecasting}

The final problem is that the agents in a
realistic model are forward looking, meaning that agents have to have
some way of forecasting what they expect to occur in the future based
on current conditions. When current conditions are determined by an
antire distribution across agents, then it is a challenging task to
translate that object into a reasonable forecast. Instead, economists
usually assume that agents use some forecasting rule based on
aggregate variables. But to be realistic, these rules can lead to
forecasting error, but should not be particularly biased
(i.e. everyone should not always forecast incorrectly in the same
way). But since behavior depends on the forecasting rule, and the bias
of the forecasting rule depends on behavior, this leads to a
``fixed-point'' problem that may involve running the solution and
simulation steps multiple times, adding to the computational burden.

\subsubsection{Computational Performance}

Despite these challenges, many of the computations involved in solving these
models are highly parallelizable, and should therefore yield massive
speedup relative to a serial computation. In particular, solving and
simulating these models often requires the type of repeated simple
calculation that is perfect for GPU computation. This should allow not
only for convenience allowed by the reduction in wait times, but much
more important, allow for more complex models or more accurate
solutions to become tractable.

\end{document}
