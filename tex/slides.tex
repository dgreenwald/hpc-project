
\documentclass[handout]{beamer}
% \documentclass{beamer}

\usetheme{Frankfurt}
\usecolortheme{seahorse}

\usepackage{amsmath}
\usepackage{amssymb}
\usepackage{amsthm}
\usepackage{amsfonts}
% \usepackage{changepage}
\usepackage{mathtools}
\usepackage{graphicx}
\usepackage[final]{pdfpages}

\newcommand{\sumn}{\sum_{i=1}^{n}}
\newcommand{\sumt}{\sum_{t=0}^\infty}
\newcommand{\fsum}{\frac{1}{n} \sum_{i=1}^n}
\newcommand{\prodn}{\prod_{i=1}^{n}}
\newcommand{\intf}{\int_{-\infty}^{\infty}}
\newcommand{\intz}{\int_0^\infty}
\newcommand{\limf}{\lim_{n\to \infty}}
\newcommand{\N}{\mathbb{N}}
\newcommand{\R}{\mathbb{R}}
\newcommand{\E}{\mathbb{E}}
\newcommand{\Z}{\mathbb{Z}}
\newcommand{\C}{\mathbb{C}}
\newcommand{\bp}{\mathbb{P}}
\newcommand{\cf}{\mathcal{F}}
\newcommand{\Var}{\text{Var}}
\newcommand{\Cov}{\text{Cov}}
\newcommand{\topr}{\xrightarrow{  p  }}
\newcommand{\tod}{\xrightarrow{  d  }}
\newcommand{\blambda}{\bar{\lambda}}
\newcommand{\htheta}{\hat{\theta}}
\newcommand{\eps}{\varepsilon}
\newcommand{\tick}{[<+->]}
\newcommand{\cond}{ \; \Bigr| \; }
\newcommand{\maxtil}{ \widetilde{\max} }

\allowdisplaybreaks[1]

\theoremstyle{definition}
\newtheorem{thm}{Theorem}
% \newtheorem*{claim}{Claim}
\newtheorem{prop}[thm]{Proposition}
\newtheorem*{defn}{Definition}
% \newtheorem*{cor}{Corollary}
% \newtheorem*{ex}{Example}
% \newtheorem*{exer}{Exercise}
% \newtheorem{lem}[thm]{Lemma}
% \newtheorem{ob}{Observation}
% \newtheorem*{fact}{Fact}

% Outline:

% 1. Economic/computational motivation.

% a. Borrowing constraints, uninsurable income risk, market incompleteness are important for understanding economic behavior.
% b. But in this case, macroeconomic (aggregate) outcomes depend on the entire distribution of agents.
% c. Need to solve numerically, typically by simulating the behavior of large numbers of agents, and setting prices to ensure market clearing.
% d. Computationally expensive problem, well-suited to parallelization.
% e. Solution of optimal policy problem also computationally expensive well-suited to parallelization.

% 2. Outline of problem.

% a. Agent's problem
% b. State space
% c. Stochastic processes
% d. Assumed values for q (qbar)
% e. Endogenous grid method.

% 3. Walk through solution.

% a. Solution objects: functions as arrays.
% b. Work group: Nx_loc x 1 x Ns
% c. Check if done.
% d. Evaluating expectations/bilinear interpolation (local memory).
% e. Endogenous grid.
% f. Return to good grid, linear interpolation/bisection (local memory).

% 4. Walk through simulation.

% a. Objects: simulations for many agents.
% b. Simulate exogenous processes on the host.
% c. Evaluate agents' policy using bilinear interpolation.
% d. Add asset holdings through reduction (local memory)
% e. Iterate to convergence on q in each period.
% f. Update and repeat.

% 5. Overall loop to convergence over qbar

\title{Agent-Based Economic Models in OpenCL}
\author{Dan Greenwald and Kevin Mullin}
\date{\today}

\begin{document}

\small

\frame{\titlepage}

\section{Introduction}

\begin{frame}
  \frametitle{Introduction}
  \begin{itemize}[<+->]
  \item Traditional macroeconomics assumes perfect insurance and asset markets to make sure that economic activity only depends on aggregate variables.
  \item However, borrowing constraints, uninsurable income risk, and market incompleteness are important features for understanding economic behavior.
  \item Relaxing these simplifying assumptions is difficult, because economic activity now depends on the entire distribution of agents' states --- requires large scale simulations to solve.
  \item Solution of optimal policy and simulation can be computationally expensive problems in this type of model, but are well suited to parallelization.
  \item Huge performance improvements on the GPU using OpenCL.
  \end{itemize}
\end{frame}

\section{Model}

\begin{frame}
  \frametitle{Agent's Problem}
  \begin{itemize}[<+->]
  \item Each infinitely-lived agent (indexed by $i$) maximizes
    \[ V(\theta_{it}) = \E_t \sum_{j=0}^\infty \beta^j u(c_{i,t+j}(\theta_{i,t+j})) \]
    at each time $t$.
    \begin{itemize}[<+->]
    \item $\theta_{i,t+j}$: state of the world at time $t+j$.
    \item $c_{i,t+j}$: state-contingent consumption plan at time $t+j$.
    \item $\beta$: discount factor (patience).
    \item $u$: utility function (enjoyment of consumption).
    \item $\E_t$: conditional expectation given time $t$ information.
    \end{itemize}
  \end{itemize}
\end{frame}

\begin{frame}
  \frametitle{Environment}
  \begin{itemize}[<+->]
  \item The economy can either be in a state of recession $(z_t = 0)$, or expansion $(z_t = 1)$.
  \item Agents can either be unemployed $(e_{it} = 0)$, or employed $(e_{it} = 1)$.
  \item Agents earn labor income when employed, and unemployment benefits when unemployed. Wages are higher in expansions than recessions.
    \begin{itemize}
    \item Total income given by $y(z_t, e_{it})$.
    \end{itemize}
  \item $z_t$ and $e_{it}$ follow a joint Markov chain for each agent, with transition matrix $P$.
  \end{itemize}
\end{frame}

\begin{frame}
  \frametitle{Asset Space}
  \begin{itemize}[<+->]
  \item Agents can save and borrow from each other risk-free.
  \item Can think of this as agents holding long (positive) and short (negative) positions in a bond ($b_{it}$) that you buy for some price $(q_t)$ today, and which pays 1 unit of consumption next period.
  \item Interest rate (equivalently, bond price) is set so that the market clears in each period (total saving equals total borrowing).
  \item Each agent has the same borrowing limit $-\bar{b}$.
  \end{itemize}
\end{frame}

\begin{frame}
  \frametitle{Optimality}
  \begin{itemize}[<+->]
  \item Each agent's policy at time $t$ depends on current wealth ($x_{it}$), current bond price $(q_t)$, and current state $s_{it} = (z_t, e_{it})$.
  \item Optimality condition
    \[ q_t u'(c_t(x_{it}, q_t, s_{it})) \ge \beta \E_t u'(c_{t+1}(x_{i,t+1}, q_{t+1}, s_{i,t+1})) \]
    with equality as long as $b_{it} > -\bar{b}$.
  \item If $u$ is well behaved, then solution $\{c_t\}$ is uniquely defined by this equation.
  \item Problem: you don't know the distribution of $q_{t+1}$.
  \item Solution: have agents assume that $q_{t+1} = \tilde{q}(z_{t+1})$. We will want to choose $\tilde{q}(z_{t+1})$ so that it approximates $q_{t+1}$ well (or is at least unbiased).
  \end{itemize}
\end{frame}

\section{Computational Method: Solution}

\begin{frame}
  \frametitle{Solution}
  \begin{itemize}[<+->]
  \item Want to solve for optimal consumption policy $c(x, q, s)$.
  \item Since $x$ and $q$ are continuous variables, this is an infinite-dimensional object, so approximate on a set of gridpoints $(\bar{x}_1, \ldots, \bar{x}_{N_x})$, $(\bar{q}_1, \ldots, \bar{q}_{N_q})$, and use bilinear interpolation between gridpoints.
  \item Strategy: initialize $c^0$ with some reasonable starting point (i.e., consume all assets), and iterate on
    \[ q u'(c^{n+1}(x_i, q, s_i)) \ge \beta \sum_{s_i'} P(s_i, s_i') u'(c^n(x_i', q', s_i')) \]
    until $\max(c^{n+1} - c^n) < \eps$.
  \end{itemize}
\end{frame}


\begin{frame}
  \frametitle{Solution Algorithm}
  \begin{itemize}[<+->]
  \item Use workgroups of size $(K_x, 1, N_s)$ (recall $N_s = 4$).
  \item Global index $(i, j, k)$.
  \item Step 1: calculate $c^n(\bar{x}_i, \tilde{q}(\bar{s}_k), \bar{s}_k)$ in local memory using bilinear interpolation (easy because grid is known).
  \item Step 2: calculate
    \[ \E_t c^n | \bar{s}_k = \sum_m P(\bar{s}_k, \bar{s}_m) c^n(\bar{x}_i, \tilde{q}(\bar{s}_m), \bar{s}_m) \]
    using previous results.
  \end{itemize}
\end{frame}

\begin{frame}
  \frametitle{Solution Algorithm}
  \begin{itemize}[<+->]
  \item Step 3: invert optimality condition to obtain 
  \end{itemize}
\end{frame}

\end{document}
