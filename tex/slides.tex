
\documentclass[handout]{beamer}
% \documentclass{beamer}

\usetheme{Frankfurt}
\usecolortheme{seahorse}

\usepackage{amsmath}
\usepackage{amssymb}
\usepackage{amsthm}
\usepackage{amsfonts}
% \usepackage{changepage}
\usepackage{mathtools}
\usepackage{graphicx}
\usepackage[final]{pdfpages}

\newcommand{\sumn}{\sum_{i=1}^{n}}
\newcommand{\sumt}{\sum_{t=0}^\infty}
\newcommand{\fsum}{\frac{1}{n} \sum_{i=1}^n}
\newcommand{\prodn}{\prod_{i=1}^{n}}
\newcommand{\intf}{\int_{-\infty}^{\infty}}
\newcommand{\intz}{\int_0^\infty}
\newcommand{\limf}{\lim_{n\to \infty}}
\newcommand{\N}{\mathbb{N}}
\newcommand{\R}{\mathbb{R}}
\newcommand{\E}{\mathbb{E}}
\newcommand{\Z}{\mathbb{Z}}
\newcommand{\C}{\mathbb{C}}
\newcommand{\bp}{\mathbb{P}}
\newcommand{\cf}{\mathcal{F}}
\newcommand{\Var}{\text{Var}}
\newcommand{\Cov}{\text{Cov}}
\newcommand{\topr}{\xrightarrow{  p  }}
\newcommand{\tod}{\xrightarrow{  d  }}
\newcommand{\blambda}{\bar{\lambda}}
\newcommand{\htheta}{\hat{\theta}}
\newcommand{\eps}{\varepsilon}
\newcommand{\tick}{[<+->]}
\newcommand{\cond}{ \; \Bigr| \; }
\newcommand{\maxtil}{ \widetilde{\max} }

\allowdisplaybreaks[1]

\theoremstyle{definition}
\newtheorem{thm}{Theorem}
% \newtheorem*{claim}{Claim}
\newtheorem{prop}[thm]{Proposition}
\newtheorem*{defn}{Definition}
% \newtheorem*{cor}{Corollary}
% \newtheorem*{ex}{Example}
% \newtheorem*{exer}{Exercise}
% \newtheorem{lem}[thm]{Lemma}
% \newtheorem{ob}{Observation}
% \newtheorem*{fact}{Fact}

% Outline:

% 1. Economic/computational motivation.

% a. Borrowing constraints, uninsurable income risk, market incompleteness are important for understanding economic behavior.
% b. But in this case, macroeconomic (aggregate) outcomes depend on the entire distribution of agents.
% c. Need to solve numerically, typically by simulating the behavior of large numbers of agents, and setting prices to ensure market clearing.
% d. Computationally expensive problem, well-suited to parallelization.
% e. Solution of optimal policy problem also computationally expensive well-suited to parallelization.

% 2. Outline of problem.

% a. Agent's problem
% b. State space
% c. Stochastic processes
% d. Assumed values for q (qbar)
% e. Endogenous grid method.

% 3. Walk through solution.

% a. Solution objects: functions as arrays.
% b. Work group: Nx_loc x 1 x Ns
% c. Check if done.
% d. Evaluating expectations/bilinear interpolation (local memory).
% e. Endogenous grid.
% f. Return to good grid, linear interpolation/bisection (local memory).

% 4. Walk through simulation.

% a. Objects: simulations for many agents.
% b. Simulate exogenous processes on the host.
% c. Evaluate agents' policy using bilinear interpolation.
% d. Add asset holdings through reduction (local memory)
% e. Iterate to convergence on q in each period.
% f. Update and repeat.

% 5. Overall loop to convergence over qbar

\title{Agent-Based Economic Models in OpenCL}
\author{Dan Greenwald and Kevin Mullin}
\date{\today}

\begin{document}

\small

\frame{\titlepage}

\section{Introduction}

\begin{frame}
  \frametitle{Introduction}
  \begin{itemize}[<+->]
  \item Traditional macroeconomics assumes perfect insurance and asset markets to make sure that economic activity only depends on aggregate variables.
  \item However, borrowing constraints, uninsurable income risk, and market incompleteness are important features for understanding economic behavior.
  \item Relaxing these simplifying assumptions is difficult, because economic activity now depends on the entire distribution of agents' states --- requires large scale simulations to solve.
  \item Solution of optimal policy and simulation can be computationally expensive problems in this type of model, but are well suited to parallelization.
  \item Our example yields substantial performance improvements on the GPU using OpenCL.
  \end{itemize}
\end{frame}

\section{Model}

% \begin{frame}
%   \frametitle{Agent's Problem}
%   \begin{itemize}[<+->]
%   \item Each infinitely-lived agent (indexed by $i$) maximizes
%     \[ V(\theta_{it}) = \E_t \sum_{j=0}^\infty \beta^j u(c_{i,t+j}(\theta_{i,t+j})) \]
%     at each time $t$.
%     \begin{itemize}[<+->]
%     \item $\theta_{i,t+j}$: state of the world at time $t+j$.
%     \item $c_{i,t+j}$: state-contingent consumption plan at time $t+j$.
%     \item $\beta$: discount factor (patience).
%     \item $u$: utility function (enjoyment of consumption).
%     \item $\E_t$: conditional expectation given time $t$ information.
%     \end{itemize}
%   \end{itemize}
% \end{frame}

\begin{frame}
  \frametitle{Model Environment}
  \begin{itemize}[<+->]
  \item The economy has a macro state $z$ that can correspond to either recession $(z = 0)$, or expansion $(z = 1)$.
  \item Infinitely-lived agents (indexed by $i$) can either be unemployed $(e_i = 0)$, or employed $(e_i = 1)$.
  \item Define $s_i = (z, e_i)$, and assume that $s_i$ follows a Markov chain with transition matrix $P$.
  \item Agents' income depends on both macro state and employment state, denote by $y(s_i)$.
  \end{itemize}
\end{frame}

\begin{frame}
  \frametitle{Borrowing and Saving}
  \begin{itemize}[<+->]
  \item Agents can save and borrow from each other using one-period loans.
  \item Equivalently: agents hold positive or negative positions in a one-period risk free bond, denoted by $b_i$.
  \item You buy (sell) this bond at market price $q$ today, and receive (pay) one unit of consumption next period.
  \item At equilibrium, $q$ must be set so that the market clears in each period (total saving equals total borrowing).
  \item Each agent has an identical borrowing limit $b_i \ge -\bar{b}$.
  \end{itemize}
\end{frame}

% \begin{frame}
%   \frametitle{Optimality}
%   \begin{itemize}[<+->]
%   \item Each agent's policy at time $t$ depends on current wealth ($x_{it}$), current bond price $(q_t)$, and current state $s_{it} = (z_t, e_{it})$.
%   \item Optimality condition
%     \[ q_t u'(c_t(x_{it}, q_t, s_{it})) \ge \beta \E_t u'(c_{t+1}(x_{i,t+1}, q_{t+1}, s_{i,t+1})) \]
%     with equality as long as $b_{it} > -\bar{b}$.
%   \item If $u$ is well behaved, then solution $\{c_t\}$ is uniquely defined by this equation.
%   \item Problem: you don't know the distribution of $q_{t+1}$.
%   \item Solution: have agents assume that $q_{t+1} = \tilde{q}(z_{t+1})$. We will want to choose $\tilde{q}(z_{t+1})$ so that it approximates $q_{t+1}$ well (or is at least unbiased).
%   \end{itemize}
% \end{frame}

\begin{frame}
  \frametitle{Optimality Conditions}
  \begin{itemize}[<+->]
  \item Need to determine optimal policy of the agent ($c_i$ and $b_i$) as functions $(x_i, q, s_i)$, where $x_i$ is starting wealth (from previous bonds).
  \item Optimal policy uniquely determined by
    \begin{equation*}
      q u'(c_i) \ge \beta \E \left[ u'(c_i') | s_i \right]
    \end{equation*}
    \begin{itemize}[<+->]
    \item Must hold with equality for $b_i > -\bar{b}$ (complementary slackness).
    \item $\beta$ is the discount factor (patience).
    \item $\E$ is the expectation operator.
    \item Primes represent next values (and derivatives --- sorry!).
    \end{itemize}
  \end{itemize}
\end{frame}

\section{Solution Algorithm}

\begin{frame}
  \frametitle{Solution Algorithm}
  \begin{itemize}[<+->]
  \item Want to solve for optimal consumption policy $c(x, q, s)$.
  \item Since $x$ and $q$ are continuous variables, this is an infinite-dimensional object, so approximate on a set of gridpoints $(\bar{x}_0, \ldots, \bar{x}_{N_x-1})$, $(\bar{q}_1, \ldots, \bar{q}_{N_q-1})$, and use bilinear interpolation between gridpoints.
  \item Strategy: initialize $c^0$ with some reasonable starting point (i.e., consume all assets), and iterate on
    \begin{equation}
      \label{eq:foc}
      q u'(c^{n+1}(x_i, q, s_i)) \ge \beta \sum_{s_i'} P(s_i, s_i') u'(c^n(x_i', q', s_i'))
    \end{equation}
    until $\max( | c^{n+1} - c^n | ) < \eps$.
    %% \item This means leaving $(c^n, c^{n+1})$ arrays in device memory at all times, and updating each step.
  \end{itemize}
\end{frame}


\begin{frame}
  \frametitle{Endogenous Grid Method}
  \begin{itemize}[<+->]
  \item Simple method: for each current gridpoint $(x_i, q, s_i)$, solve \eqref{eq:foc} using a nonlinear equation solver.
  \item This is slow, requires many guesses and function evaluations for each gridpoint.
  \item Better: start on grid of bond holdings, exploit the fact that $x_i' = b_i + y(s_i')$, and then invert \eqref{eq:foc} using
    \[ c_i^* = (u')^{-1} \left\{ \beta q^{-1} \sum_{s'} P(s,s') u'(c^n(b_i + y(s_i'), \tilde{q}', s')) \right\} \]
  \item This defines a $(b_i, c_i^*)$ correspondence, but we can recover starting wealth $x_i^*$ using the budget constraint $x_i^* + y(s_i) = c_i^* + b_i$, to obtain correspondence $(x_i^*, c_i^*)$.
  \item However, $x_i^*$ does not fall on our original grid, but is an output of the algorithm, so the set of $x_i^*$ is called an \emph{endogenous} grid.
  \end{itemize}
\end{frame}

\begin{frame}
  \frametitle{Endogenous Grid Steps}
  \begin{itemize}[<+->]
  \item Use workgroups of size $(K_x, 1, N_s)$, where $K_x$ is some local $x$ size.
  \item Step 1: calculate $c^n(b_i + y(s_i'), \tilde{q}, s_i')$ using bilinear interpolation (easy because grid is known) and store in local memory.
  \item Step 2: calculate
    \[ \E \left[ c^n(b_i + y(s_i'), \tilde{q}, s_i') | s_i \right] = \sum_m P(s_i, s_i') c^n(x_i, \tilde{q}, s') \]
    using previous results.
  \item Step 3: invert \eqref{eq:foc} to obtain $(c_i^*, x_i^*)$ correspondence given $(q, s_i)$.
  \end{itemize}
\end{frame}

\begin{frame}
  \frametitle{Recovering Original Grid}
  \begin{itemize}[<+->]
  \item No good if each $(q, s_i)$ associated with unique grid, need to get back from endogenous $x_i^*$ grid onto original $x_i$ grid.
  \item For each $x_i$ gridpoint, search to find corresponding bin on $x_i^*$ grid, and perform linear interpolation to find $c(x_i, q, s_i)$.
  \item Problem: the relevant $x_i^*$ points are distributed across work groups.
  \item Solution: load the entire $x_i$ grid one $(K_x \times 1)$ sized block at a time into each work group.
  \item Each work item takes one $x_i$ from this block, and checks if it falls in that group's $x_i^*$ grid.
  \item If so, search for exact bin (using bisection) and interpolate.
  \item Need to overlap the work groups so that every point will fall into one of these intervals.
  \end{itemize}
\end{frame}

\begin{frame}
  \frametitle{Data Transfer and Convergence}
  \begin{itemize}[<+->]
  \item To avoid read/write delays, keep arrays for current and previous $c$ array on the device at all times.
  \item After each iteration, check (on device, in parallel) whether $| c^{n+1} - c^n | \ge \eps$ for each work item. If so, not done!
  \item No global write issues because we only care if \emph{any} points are too far.
  \item Only data transfer between host and device on each iteration is flag indicating convergence, arrays written/read only once.
  \end{itemize}
\end{frame}

% \begin{frame}
%   \frametitle{Recovering Original Grid}
%   \begin{itemize}[<+->]
%   \item Each work item takes one $x_i$ from this block, and checks if it falls in that group's $x_i^*$ grid.
%   \item If so, search for exact bin (using bisection) and interpolate.
%   \item Need to overlap the work groups so that every point will fall into one of these intervals.
%     \begin{itemize}[<+->]
%     \item Work group 1: $(\bar{x}_0,\ldots, \bar{x}_{K_x - 1})$
%     \item Work group 2: $(\bar{x}_{K_x - 1}, \ldots, \bar{x}_{2 K_x - 2})$
%     \item etc...
%     \end{itemize}
%   \end{itemize}
% \end{frame}

\section{Simulation}

\begin{frame}
  \frametitle{Simulation}
  \begin{itemize}[<+->]
  \item Solution results from the previous section tell the agents what to do given $(x_i, q, s_i)$.
  \item Apply solution to large-scale simulation (many agents, many periods) to obtain aggregate results.
  \item Main obstacle: need to solve for the unique $q$ that clears the bond market in \emph{every} period.
  \end{itemize}
\end{frame}

\begin{frame}
  \frametitle{Simulation Algorithm}
  \begin{itemize}[<+->]
  \item Each period, 1-d parallelization using work groups of $K_{SIM}$ agents.
  \item Given a guess for $q$, random draws of $s_i$, and the $x_i$ implied by past behavior, solve for $c_i$ and $b_i$.
  \item Add $b_i$ across agents using a reduction algorithm.
    \begin{itemize}[<+->]
    \item Step 1: to add all assets in a work group.
    \item Step 2: kernel to add assets across work groups.
    \end{itemize}
  \item Check total assets and adjust $q$ accordingly (bisection).
  \item Iterate until market clearing, then move to next period.
  \end{itemize}
\end{frame}

\section{Meta Routine}

\begin{frame}
  \frametitle{``Meta'' Routine}
  \begin{itemize}[<+->]
  \item All the previous steps assumed a guess for $\tilde{q}(z)$, the average bond price in each macro ($z$) state.
  \item Want these expectations to be unbiased.
  \item Starting with some guess for $\tilde{q}$, run the entire algorithm, and calculate sample means of bond prices in each $z$ state.
  \item If sample means match $\tilde{q}$ within tolerance, you are done.
  \item Otherwise, restart the routine using the previous sample means as $\tilde{q}$.
  \end{itemize}
\end{frame}

\section{Results}

\begin{frame}
	\frametitle{Sample Paths}
%	\begin{itemize}[<+->]
%	\item Sample paths from simulation:
%	\end{itemize}
	\includegraphics[width=\textwidth]{samplepaths}
\end{frame}

\begin{frame}
  \frametitle{Performance}
  \begin{itemize}[<+->]
  \item Substantial speedup on the GPU relative to CPU, more so for solution algorithm than for simulation algorithm.
  \item Roughly 1-6x speedup for the simulation algorithm depending on scale.
  \item Roughly 8-1000x speedup for the solution algorithm depending on scale.
  \item Diminishing returns to increasing $N_x$ on GPU.
  \item Roughly linear in $N_t$, $N_q$ on GPU.
  \item Timing not sensitive to increases in $N_{SIM}$ on GPU.
  \end{itemize}
\end{frame}
  
\begin{frame}
  \frametitle{Timings on NVIDIA Tesla M2070 GPU}
	\includegraphics[width=\textwidth]{timings.pdf}
\end{frame}

\begin{frame}
  \frametitle{Timings Per Gridpoint on NVIDIA M2070 Tesla GPU}
	\includegraphics[width=\textwidth]{timings_pergrid.pdf}
\end{frame}

\begin{frame}
  \frametitle{Timings on Intel Xeon CPU}
	\includegraphics[width=\textwidth]{timings_cpu.pdf}
\end{frame}

\begin{frame}
  \frametitle{Timings Per Gridpoint on Intel Xeon CPU}
	\includegraphics[width=\textwidth]{timings_pergrid_cpu.pdf}
\end{frame}

\begin{frame}
  \frametitle{Speedup: Ratio of CPU Timing to GPU Timing}
	\includegraphics[width=\textwidth]{speedup.pdf}
\end{frame}

\begin{frame}
  \frametitle{Tricks We Learned}
  \begin{itemize}[<+->]
  \item Replace constants with preprocessor macros.
  \item Similarly, re-use variables using macros, by assigning multiple names to the same object.
  \item Allocate local memory using clSetKernelArg
  \item \texttt{constant} global memory is good --- but there is a limit on how much of this memory you can use on the GPU!
  \end{itemize}
\end{frame}

\end{document}
