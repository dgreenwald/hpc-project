
\documentclass[a4paper,12pt]{article}

\usepackage{amsmath}
\usepackage{amssymb}
\usepackage{amsthm}
\usepackage{amsfonts}
\usepackage{changepage}
\usepackage{mathtools}
\usepackage{booktabs}
\usepackage{caption}

\newcommand{\sumn}{\sum_{i=1}^{n}}
\newcommand{\sumt}{\sum_{t=0}^\infty}
\newcommand{\sumj}{\sum_{j=-\infty}^\infty}
\newcommand{\fsum}{\frac{1}{n} \sum_{i=1}^n}
\newcommand{\prodn}{\prod_{i=1}^{n}}
\newcommand{\intf}{\int_{-\infty}^{\infty}}
\newcommand{\intz}{\int_0^\infty}
\newcommand{\limf}{\lim_{n\to \infty}}
\newcommand{\N}{\mathbb{N}}
\newcommand{\R}{\mathbb{R}}
\newcommand{\E}{\mathbb{E}}
\newcommand{\Z}{\mathbb{Z}}
\newcommand{\C}{\mathbb{C}}
\newcommand{\bp}{\mathbb{P}}
\newcommand{\F}{\mathcal{F}}
\newcommand{\Var}{\text{Var}}
\newcommand{\Cov}{\text{Cov}}
\newcommand{\Corr}{\text{Corr}}
\newcommand{\topr}{\xrightarrow{  p  }}
\newcommand{\tod}{\xrightarrow{  d  }}
\newcommand{\blambda}{\bar{\lambda}}
\newcommand{\htheta}{\hat{\theta}}
\newcommand{\hbeta}{\hat{\beta}}
\newcommand{\hmu}{\hat{\mu}}
\newcommand{\hF}{\hat{F}}
\newcommand{\sss}{\subsubsection*}
\newcommand{\simiid}{\stackrel{\text{iid}}{\sim}}
\newcommand{\eqas}{\stackrel{\text{a.s.}}{=}}
\newcommand{\eps}{\varepsilon}
\newcommand{\re}{\text{Re}}
\newcommand{\im}{\text{Im}}

\numberwithin{equation}{section}

\theoremstyle{definition}
\newtheorem{thm}{Theorem}
\newtheorem{claim}{Claim}
\newtheorem{prop}[thm]{Proposition}
\newtheorem{defn}{Definition}
\newtheorem{cor}{Corollary}
\newtheorem{ex}{Example}
\newtheorem{exer}{Exercise}
\newtheorem{lem}[thm]{Lemma}
\newtheorem{ob}{Observation}
\newtheorem{fact}{Fact}

\allowdisplaybreaks[1]
\linespread{1.6}
% \linespread{1.3}

\begin{document}

\small
% \footnotesize

\setlength\voffset{-0.75 in}

\changepage{1.5 in}{1 in}{0 in}{-0.5 in}{0 in}{0 in}{0 in}{0 in}{0 in}

% Description of the problem

Our project is to simulate the macroeconomic behavior of an economy
with incomplete asset markets using OpenCL on a GPU. A complete
description of the environment and the algorithm that we use can be
found in Sections [] through []. In general, the model is based on
that of Krusell and Smith (1998) adapted so that a risk-free bond is
traded, and there is no capital in the economy.

The problem contains two major sections. One is to solve for the
agent's optimal consumption policy given his or her current
state. This involves interating a functional operation on function
approximations defined on a grid. Since the calculations for each
gridpoint are independent, the algorithm is highly
parallelizable. Each iteration involves solving an equation derived
from the model's optimality conditions. This can be done without
resorting to use of a nonlinear equation solver by use of the
``endogenous grid method.'' The main computational challenge is to
move back from the endogenous grid to the original grid, because it
involves interpolation on an unknown grid, of which only a portion is
held by each work-group.

The second section is to use the optimal policies to simulate the
economy for a large number of agents over a large number of periods,
in order to obtain a simulation of macroeconomic behavior. This type
of simulation could be used to perform experiments, like studying the
effects of a policy change. Alternatively, this type of simulation
could be used for parameter estimation, which would seek to minimize
the distance between simulated moments of macroeconomic variables and
their counterparts in the actual data. The main computational
challenge in this step is to set bond prices exactly so that markets
clear (i.e. so that total saving equals total borrowing), which
involves simulating the economy in each period for various guesses of
the bond price until market clearing is attained.

Throughout the above steps, the agents use a forecasting rule to
generate their expectations of the bond price in the future, based on
the future macroeconomic state. In order that this expectation be
unbiased, the above algorithm is iterated using guesses of the policy
rule until the forecasting rule is within tolerance of the relevant
sample means.

% Scale of the problem we are aiming at

The scale of the project that we are aiming at for the solution
portion of the model is to use grids of between a few hundred and a
few thousand gridpoints each for the continuous variables $(x_i,
q)$. This means approximating the optimal policy function on $4 N_x
N_q$ gridpoints.

At the high end, the scale of the solution step is limited by the
available amount of memory for GPU computation, since two approximate
functions must be stored on the device at all times, taking up $32 N_x
N_q$ bytes of space at double precision. At the low end, the ability
to parallelize across gridpoints makes this procedure economical for
nearly any scale.

The scale of the project that we are aiming at for the simulation
portion of the model is to use a few thousand agents over thousands of
periods, for example, 5,000 agents over 12,000 periods (of which 2,000
is burn-in).

At the high end, the limitation is again the available amount of
memory for GPU computation, as the program is currently written, since
the simulation output arrays must hold $24 N_{sim} N_t$ bytes on the
device. At the low end, since parallelization can only occur across
agents in the simulation, and not across time, this algorithm may not
yield speedup for very small values of $N_{sim}$.

% 

\end{document}